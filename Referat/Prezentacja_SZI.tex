\documentclass[a4paper,10pt]{beamer}
\usepackage[utf8]{inputenc}
\usepackage{polski}
\usepackage[OT4,T1]{fontenc}
\usepackage{amsmath}
\usepackage{amsthm}
\usepackage{graphicx}
\usepackage{ dsfont }
\usepackage{ amssymb }
\usepackage{enumerate}
\usepackage{tikz}

\usetheme{Warsaw}
\usecolortheme{beaver}

\newtheorem{defi}{Definicja}[subsection]
\newtheorem{uw}{Uwaga}[subsection]
\newtheorem{cel}{Cel}[subsection]
\newtheorem{tw}{Twierdzenie}[subsection]
\newtheorem{lem}{Lemat}[subsection]
\newtheorem{przyk}{Przykład}[subsection]
\newtheorem{alg}{Algorytm}[subsection]

\date{15 czerwca 2015}
\title{Inteligentny kelner}
\author[A. Bohonos, D. Demski, A. Limiszewska, A. Mieldzioc]{Andrzej Bohonos, Dominik Demski, Agnieszka Limiszewska, Adam Mieldzioc}

\begin{document}
		\begin{frame}
			\titlepage
		\end{frame}
		\begin{frame}{Agenda}
			\tableofcontents
		\end{frame}
		
		\section{Opis projektu}
		\begin{frame}
			Celem naszego projektu jest zasymulowanie działania kelnera pracującego w restauracji. Kelner realizuje następujące zadania:
			\begin{itemize}
				\item ustala kolejność przygotowania zamówionych posiłków,
				\item układa gotowe dania na tacy, 
				\item decyduje, kiedy ma iść roznieść zamówione posiłki,
				\item wybiera najbardziej optymalną ścieżkę rozdania tych posiłków. 
			\end{itemize}
		\end{frame}
		\begin{frame}{Podział pracy}
			\begin{itemize}
				\item Agnieszka - ustalanie kolejności przygotowywania posiłków (drzewa decyzyjne),
				\item Adam - decydowanie o momencie roznoszenia posiłków przez kelnera (sieci neuronowe),
				\item Andrzej - znajdowanie najefektywniejszej ścieżki podczas roznoszenia posiłków (przeszukiwanie przestrzeni stanów),
				\item Dominik - znajdowanie najefektywniejszej ścieżki podczas roznoszenia posiłków (algorytmy genetyczne).
			\end{itemize}
		\end{frame}
		\begin{frame}{Przyjęte założenia}
			\begin{itemize}
				\item Mapa o wymiarach $21\times17$ składająca się z kwadratowych pól. 
				\item Pole może reprezentować pojedynczy stolik, krzesło lub puste miejsce.
				\item Każdy stolik posiada unikalny numer.
				\item Przy każdym uruchomieniu losowana jest jedna z trzech map.
				\item Zamówienia są składane losowo na podstawie przygotowanego menu.
				\item Kelner odbiera przygotowane posiłki w lewym górnym rogu planszy.
				\item Kelner realizując zamówienie podchodzi do jednego z cztrech rogów stolika.
				
			\end{itemize}
		\end{frame}
		
		\begin{frame}{Reprezentacja wiedzy - system ram}
			Stworzenie systemu ram zrealizowaliśmy za pomocą paradygmatu programowania obiektowego. Instancje poszczególnych klas odpowiadają obiektom świata rzeczywistego.
		\end{frame}
		\begin{frame}{Reprezentacja wiedzy o posiłkach}
			Klasa Meal - reprezentuje posiłek znajdujący się w menu:
			\begin{itemize}
				\item name - nazwa posiłku,
				\item ingredients - składniki posiłku,
				\item type - konsystencja,
				\item temp - temperatura podania posiłku,
				\item area - pole powierzchni zajmowanej na tacy,
				\item time - czas przygotowania. 
			\end{itemize}
		\end{frame}
		\begin{frame}{Reprezentacja wiedzy o zamówieniach}
			Klasa Order - reprezentuje zamówienie złożone przez klienta:
			\begin{itemize}
				\item meal - zamawiany posiłek,
				\item tableNumber - numer stolika, który złożył zamówienie,
				\item orderTime - czas od złożenia zamówienia,
				\item prepareTime - czas przygotowania posiłku,
				\item VIP - priorytet zamówienia,
				\item liquid - czy zamówionym posiłkiem jest napój.
			\end{itemize}
		\end{frame}
		\begin{frame}{Zarządzanie zamówieniami}
			Klasa OrderService - przechowuje informacje o zamówieniach:
			\begin{itemize}
				\item menu - lista posiłków,
				\item orders - lista złożonych zamówień,
				\item currentCreatingMeals - lista zamówień przygotowywanych przez kuchnię,
				\item readyMeals - lista gotowych zamówień,
				\item tray - lista zamówień znajdujących się na tacy kelnera.
			\end{itemize}
		\end{frame}
		
		\section{Algorytmy genetyczne}
		\begin{frame}{Do czego zastosowałem algorytmy genetyczne?}
			Skorzystałem z biblioteki JGAP.
			\begin{defi}
				Komiwojażer musi odwiedzić wszystkie miasta z zadanego regionu i wrócić do miasta początkowego (jest to problem szukania cyklu). Wszystkie miasta są ze sobą połączone (mamy do czynienia z grafem pełnym). Mając do dyspozycji macierz odległości pomiędzy poszczególnymi miastami, należy znaleść cykl o najmniejszym koszcie, przy czym każde miasto nie może być odwiedzone więcej niż jeden raz.
			\end{defi}
			
		\end{frame}
		\begin{frame}{Problem komiwojażera w aplikacji}
			\begin{enumerate}
				\item miasta = stoły
				\item komiwojażer = kelner
			\end{enumerate}
		\end{frame}
		\begin{frame}{Kolejne elementy algorytmu genetycznego}
			\begin{alg}
				\begin{enumerate}
					\item Populacja
					\item Krzyżowanie
					\item Mutacja
					\item Ocena osobników
					\item Selekcja
				\end{enumerate}
			\end{alg}
		\end{frame}
		\begin{frame}{Populacja}
			\begin{enumerate}
				\item Rozmiar populacji: 1000
				\item Ilość ewolucji: 2000
				\item Na początku osobniki tworzone są w sposób losowy - na podstawie przykładowego chromosomu.
			\end{enumerate}
		\end{frame}
		\begin{frame}{Reprezentacja chromosomu}
			Pojedynczy chromosom ma reprezentować proponowaną kolejność odwiedzania stolików. Chromosom reprezentuję jako listę pokazująca kolejność pobierania stolików do tworzonej trasy.
			
			\begin{przyk}
					Punktem odniesienia dla tej reprezentacji jest lista kolejnych stolików: 1-2-3-4-5. Pojedynczy osobnik np. 3-3-0-1-0 pokazuje w jakiej kolejności wybierane są kolejno odwiedzane stoliki. Na początku jest trójka więc pierwszym odwiedzanym stolikiem będzie stolik umieszczony na trzeciej pozycji w liście odniesienia, czyli czwórka. Czwórkę tę usuwa się z listy odniesienia (pozostają stoliki 1-2-3-5), natomiast lista odwiedzanych stolików wygląda następująco: 4.
					Kolejnym elementem osobnika jest ponownie trójka. W tej chwili na trzecim miejscu listy odniesienia jest piątka, więc kolejnym odwiedzanym stolikiem będzie stolik nr 5, a lista odniesienia będzie wyglądała następująco: 1-2-3 itd.
			\end{przyk}
		\end{frame}
		\begin{frame}{Wady i zalety takiej reprezentacji}
			Reprezentacja ta wprowadza spore zamieszanie przy przechodzeniu na reprezentację wykorzystywaną przy funcji oceny, jednak kłopoty te rekompensuje przy krzyżowaniu i mutacji. Cechą charakterystyczą tej reprezentacji jest fakt, że na i-tej pozycji jest liczba z przedziału od 0 do n-i-1 (gzie n to liczba wszystkich stolików, np. na miejscu zerowym wszystkie miejsca są jeszcze do wybrania, jest więc n - 0 możliwości, należy odjąć 1, bo lista numerowana jest od zera). Ze względu na to wymiana materiału genetycznego między dwoma osobnikami za pomocą standardowego krzyżowania x-punktowego zawsze da dopuszczalne potomstwo.
		\end{frame}
		\begin{frame}{Kod}
			\pgfdeclareimage[width=10cm,height=8cm]{chromosom}{przykladowyChromosom.png}
			\pgfuseimage{chromosom}
		\end{frame}
		\begin{frame}{Kod}
			\pgfdeclareimage[width=9cm,height=6cm]{chromosom}{zamianaNaChromosom.png}
			\pgfuseimage{chromosom}
		\end{frame}
	
		\begin{frame}{Krzyżowanie}
			35 procent populacji bierze udział w krzyżowaniu - każde krzyżowanie daje dwa nowe osobniki. Zastosowałem krzyżowanie jednopunktowe, punkt krzyżowania wybierany jest w sposób losowy. 
		\end{frame}
		\begin{frame}{Mutacja}
			Mutacja występuje, jesli wylosowana z przedziału [0, 12] liczba całkowita jest zerem, co daje prawdopodobieństwo mutacji 1/12. Taka operacja jest stosowana do każdego genu w każdym chromosomie - z wyjątkiem tych, które dopiero co powstały w wyniku krzyżowania.
		\end{frame}
			\begin{frame}{Ocena osobników - jak mierzę odległość między stolikami?}
				Wykorzystuję odległość miejską - bo tak mniej więcej będzie chodził kelner:
				\begin{defi}
					Odległość miejska (zwana również odległością taksówkową lub manhatańską):  
					\begin{equation}
					\rho (\textbf{x}_{r}, \textbf{x}_s) = \sum\limits_{i = 1}^{p} |(x_{ri} - x_{si})|
					\end{equation}
				\end{defi}
			\end{frame}
			\begin{frame}
				\pgfdeclareimage[width=11cm,height=8cm]{chromosom}{mapa.png}
				\pgfuseimage{chromosom}
			\end{frame}
			\begin{frame}
				\pgfdeclareimage[width=11cm,height=8cm]{chromosom}{macierzOdleglosci.png}
				\pgfuseimage{chromosom}
			\end{frame}
			\begin{frame}{Ocena osobników}
				\pgfdeclareimage[width=10cm,height=8cm]{chromosom}{funkcjaDopasowania.png}
				\pgfuseimage{chromosom}
			\end{frame}
		\begin{frame}{Selekcja}
			Zastosowałem domyślną implementację (domyślnie ustawiana jest klasa BestChromosomesSelector): wybranych zostaje 90 procent najlepsszych osobników, pozostałe 10 procent uzupełnia się, kopiując najlepszych (pod względem funckji dopasowania) osobników aż do wyczerpania miejsca.
		\end{frame}
			\begin{frame}{Przykład działania}
				W ramach eksperymentu stworzyłem sobie następującą sytuację:
				\begin{enumerate}
					\item Należy, wyruszając z punktu 0, odwiedzić wszystkie stoły numerowane od 1 do 5 włącznie i wrócić do punktu 0 (każdy stół musi być odwiedzony tylko raz). 
					\item Odległość między dwoma stołami to wartość bezwzględna z różnicy ich numerów (np. odległość pomiędzy stołem 1., a 3. = |1 - 3| = 2), przy czym odległość pomiędzy stołem numer 5 i punktem 0 wynosi 1.
					\item Przy tak określonym zadaniu nietrudno sprawdzić, że najkrótsza droga to: [1, 2, 3, 4, 5] albo równoważna jej [5, 4, 3, 2, 1] i jej długość wynosi: 6.
					\item Uruchomię mój algorytm 10 razy z takimi samymi ustawieniami, jakie są zaimplementowane w aplikacji Inteligentny kelner (rozmiar populacji - 1000, 2000 ewolucji).
				\end{enumerate}
			\end{frame}
		\begin{frame}{Wyniki doświadczenia}
			\pgfdeclareimage[width=12cm,height=7cm]{chromosom}{eksperymentDD.png}
			\pgfuseimage{chromosom}
		\end{frame}
		\begin{frame}{Wyniki doświadczenia - 100 powtórzeń}
		\pgfdeclareimage[width=11cm,height=8cm]{chromosom}{eksperyment100.png}
		\pgfuseimage{chromosom}
		\end{frame}
		\begin{frame}{Wyniki doświadczenia - 1000 powtórzeń}
			Algorytm wykonywał się 1, 5 godziny.
			\pgfdeclareimage[width=12cm,height=7cm]{chromosom}{eksperyment1000.png}
			\pgfuseimage{chromosom}
		\end{frame}

		

		\section{Drzewa decyzyjne}
		\begin{frame}{Drzewa decyzyjne jako model klasyfikacji}
			Przez drzewo decyzyjne rozumiemy strukturę, która ma zwykłe właściwości drzewa w znaczeniu, jaki temu słowu nadaje się w informatyce, jest więc strukturą złożoną z węzłów, z których wychodzą gałęzie prowadzące do innych węzłów lub liści, oraz z liści, z których nie wychodzą żadne gałęzie. Węzły 			odpowiadają testom przeprowadzanym na wartościach atrybutów przykładów, gałęzie odpowiadają możliwym wynikom tych testów, liście zaś etykietom kategorii.
		\end{frame}

		\begin{frame}{Implementacja}
			\pgfdeclareimage[width=11.18cm,height=3.82cm]{treeClass}{treeClass.png}
			\pgfuseimage{treeClass}
			
		\end{frame}
		
		

		\begin{frame}{Do czego wykorzystujemy drzewa decyzyjne?}
			Zamówienia są generowane w sposób losowy. Przyjęłam założenie, że mamy 5 równolegle pracujących kucharzy, a każdy z nich podejmuje decyzję o wyborze potrawy do przygotowania. Każdy z nich stosuje te same kryteria wyboru, sposób ich rozumowania jest przedstawiony w postaci drzewa. 
		\end{frame}
		
		\begin{frame}{Schemat drzewa decyzyjnego}
			\pgfdeclareimage[width=10.875cm,height=7cm]{treeDiagram}{treeDiagram.png}
			\pgfuseimage{treeDiagram}
		\end{frame}

		\begin{frame}{Zalety drzew decyzyjnych}
			Drzewa decyzyjne: 
				\begin{itemize}
						\item{Mogą reprezentować dowolnie złożone pojęcia pojedyncze lub wielokrotne, jeśli tylko ich definicje  da się wyrazić w zależności od atrybutów}
						\item{Są efektywne obliczeniowo, w najgorszym wypadku proces identyfikacji kategorii wymaga jednokrotnego przetestowania wszystkich atrybutów}
						\item {Forma reprezentacji jest czytelna dla człowieka}
						\item {Łatwo można przejść od reprezentacji drzewiastej do reprezentacji regułowej}
						\end{itemize}
		\end{frame}

		\begin{frame}{Wady drzew decyzyjnych}
			\begin{itemize}
				\item{Testuje się wartość jednego atrybutu na raz, co powoduje niepotrzebny rozrost drzewa dla danych, gdzie poszczególne atrybuty zależą od pozostałych}
				\item{Kosztowna reprezentacja alternatyw pomiędzy atrybutami - znaczny rozrost drzewa (w przeciwieństwi do reprezentacji koniunkcji, która jest zapisywana jako pojedyńcza "ścieżka")}
				\item{Drzewa decyzyjne nie stwarzają łatwej możliwości do ich inkrementacyjnego aktualizowania, algorytmy udoskonalające gotowe już drzewa poprzez zestaw nowych przykładów są bardzo złożone i zazwyczaj wynikiem jest drzewo gorszej jakości niż drzewo budowane od początku z kompletnym zestawem przykładów.}
			\end{itemize}
		\end{frame}
		
		
		\section{Sieci neuronowe}
		
		\begin{frame}
			\begin{defi}
				\textbf{Sztuczna (symulowana) sieć neuronowa} to połączona grupa sztucznych komórek nerwowych, która wykorzystuje model matematyczny w celu przetwarzania informacji.
			\end{defi}
			 Inspiracją do rozpoczęcia badań nad sztucznymi sieciami neuronowymi było działanie ludzkiego mózgu i próba odtworzenia jego niezwykłej mocy obliczeniowej. Sieci neuronowe składają się ze sztucznych neuronów, które są tutaj odpowiednikami ludzkiej komórki nerwowej. Sztuczne sieci neuronowe mają bardzo szerokie zastosowania od ekonomii począwszy, przez medycynę do techniki.
			
		\end{frame}
		\begin{frame}{Zastosowania sieci neuronowych}
				\begin{itemize}
					\item{Klasyfikacja i rozpoznawanie wzorców (np. pisma),}
					\item{Kojarzenia danych,}
					\item{Synteza mowy,}
					\item{Aproksymacja i prognozowanie.}
				\end{itemize}
		\end{frame}
		\begin{frame}
			Jest wiele różnych rodzajów sieci neuronowych i sposobów ich uczenia. W swoim projekcie zastosowałem wielowarstwową sieć neuronową. Jest to sieć jednokierunkowa (sygnał płynie od warstwy wejściowej do warstwy wyjściowej), która składa się z trzech warstw (warstwa - zbiór neuronów między którymi nie ma żadnych połączeń). Do uczenia sieci użyłem \textbf{algorytmu wstecznej propagacji błędu z członem momentum}. Jest to jeden z najpopularniejszych sposobów uczenia sieci.
		\end{frame}
		\begin{frame}{Algorytm wstecznej propagacji błędu}
			Działanie algorytmu uczenia opiera się na obliczeniu wyjścia sieci, a później propagowaniu błędów wstecz (od warstwy wyjściowej do warstwy wejściowej), ponieważ nie znamy błędów sieci w warstwach ukrytych.\\
			 Podstawą matematyczną optymalizacji wektora wag jest \textbf{metoda największego spadku}, która ma za zadanie minimalizację poniższej funkcji błędu:
			\begin{center}
				$Q(\mathbf{w}(t))=\sum_{i=1}^{N_{L}}(d_{i}^{(L)}-y_{i}^{(L)}(t))^{2}$, gdzie
			\end{center}
			\begin{itemize}
				\item{$\mathbf{w}(t)$ - wektor wag,}
				\item{$d_i$ - wartość $i$-tego indeksu wektora wzorcowego pochodzącego ze zbioru uczącego,}
				\item{$y_i$ - wyjście sieci $i$-tego neuronu warstwy wyjściowej,}
				\item{$N_L$ - liczba neuronów wyjściowych.}
			\end{itemize}
		\end{frame}
		\begin{frame}
			Poprawa wektora wag odbywa się według następującej formuły:
			\begin{center}
				$\mathbf{w}(t+1)=\mathbf{w}(t)-\eta \nabla [Q(\mathbf{w})(t)]$, 
			\end{center}
			gdzie $\eta$ to współczynnik uczenia. Powtarzamy obliczanie:  wyjścia sieci dla obserwacji uczącej, wartość funkcji błędu dla tej obserwacji i poprawiamy wektor wag. Wykonujemy powyższe działania dla każdej obserwacji ze zbioru uczącego. Wykonujemy powyższe procedury aż do momentu, w którym skumulowana wartość funkcji błędu osiągnie satysfakcjonujący dla nas poziom.
		\end{frame}
		\begin{frame}{Wady i zalety algorytmu wstecznej propagacji błędu}
			Zalety:
			\begin{itemize}
				\item jest łatwy do zrozumienia, przez co zdobył zaufanie użytkowników,
				\item ma przewagę nad innymi algorytmami, gdy założenia niezbędne do ich działania są trudne do zrealizowania lub zweryfikowania,
				\item jest metodą powolną, ale w miarę pewną - używając tego algorytmu można prosto rozwiązać prawie każde zadanie, inne algorytmy mogą rozwiązywać zadanie bardzo szybko albo całkowicie zawodzić. 
			\end{itemize}
		\end{frame}
		\begin{frame}
			Wady:
			\begin{itemize}
			\item duża liczba iteracji potrzebna do osiągnięcia minimum,
			\item konieczność wybrania dobrego współczynnika uczenia - zbyt mały spowalnia proces uczenia, a zbyt duży może prowadzić do oscylacji,
			\item wrażliwość na występowanie minimów lokalnych.
			\end{itemize}
		\end{frame}
		\begin{frame}{Momentum}
			W celu przyspieszenia procesu uczenia stosuje się \emph{momentową metodę wstecznej propagacji błędu}. Umożliwia ona bezpieczne zwiększenie efektywnego tempa uczenia oraz stabilność samego procesu. Istotą tej modyfikacji jest wprowadzenie do procesu uaktualniania wagi współczynnika bezwładności, tzw. ,,momentu'', proporcjonalnego do zmiany tej wagi w poprzedniej iteracji. Modyfikacja wag odbywa się wtedy zgodnie z niżej podanym wyrażeniem:
			\begin{center}
				$\mathbf{w}(t+1)=\mathbf{w}(t)-\eta \nabla [Q(\mathbf{w})(t)] -\alpha(\mathbf{w}(t)-\mathbf{w}(t-1)),$
			\end{center}
			gdzie $\alpha$ to wspominany współczynnik momentum. 
		\end{frame}
		\begin{frame}{Implementacja}
				\pgfdeclareimage[width=10cm,height=8cm]{BP}{Backprop1.png}
				\pgfuseimage{BP}
		\end{frame}
		\begin{frame}
			\pgfdeclareimage[width=10cm,height=8cm]{BP}{Backprop2.png}
			\pgfuseimage{BP}
		\end{frame}
		\begin{frame}
			\pgfdeclareimage[width=10cm,height=8cm]{BP}{Backprop3.png}
			\pgfuseimage{BP}
		\end{frame}
		\begin{frame}
			\pgfdeclareimage[width=10cm,height=8cm]{BP}{Backprop4.png}
			\pgfuseimage{BP}
		\end{frame}
		\begin{frame}{Zastosowanie sieci neuronowej w projekcie Inteligenty kelner}
			Stworzona sieć neuronowa odpowiada za wydawanie decyzji czy inteligenty kelner ma iść roznosić posiłki zamówione przez klientów czy poczekać, aż zostanie zrealizowane przez kuchnię kolejne zamówienie.  
		\end{frame}
		\begin{frame}{Eksperyment}
			Przeprowadziłem eksperyment dla zbioru uczącego i sieci neuronowej o 5 neuronach wejściowych, 13 neuronach ukrytych i 1 neuronie wyjściowym. Algorytm kończył działanie, gdy błąd średniokwadratowy był mniejszy od $0.1$. Współczynnik uczenia oraz człon momentu miały odpowiednio wartości: $0.2$ i $0.6$. Otrzymałem następujące wyniki:
		\end{frame}
		\begin{frame}
			\begin{center}		
				\pgfdeclareimage[width=8cm,height=8cm]{BP}{Backprop6.png}
				\pgfuseimage{BP}
			\end{center}
		\end{frame}
		\begin{frame}{Wnioski z eksperymentu}
			Z przeprowadzonych testów można wysnuć następujące wnioski, że wartość funkcji błedu oscyluje cały czas wokół tej samej liczby. Świadczy to o tym, że znaleźliśmy się w minimum lokalnym.
			\begin{itemize}
			 \item Przyczyną tego stanu rzeczy mogą być duże różnice wartości w zbiorze uczącym. Dlatego należy zawsze w tej sytuacji znormalizować dane uczące!
			\item Innym powodem może być za mały zbiór uczący, który nie odzwierciedla za dobrze własności, których wymagamy od uczonej przez nas sieci neurowej. Należy pamiętać, żeby zbiór uczący był zawsze odpowiednio duży!
			\end{itemize}
			Pomimo wszystko błąd sieci na poziomie około $10\%$ jest akceptowalnym wynikiem. W wielu przypadkach taki poziom wystarcza do poprawnego działania sieci.
		\end{frame}
		\section{Przeszukiwanie przestrzeni stanów}
		\begin{frame}{Problem poruszania się kelnera po mapie}
			Zakładamy, że kelner przygotował już tacę z zamówieniami i musi roznieść posiłki do odpowiednich stolików. Jego celem jest wybranie najkrótszej ścieżki przechodzącej przez wszystkie stoliki.
			\begin{enumerate}
				\item Problem pierwszy: znalezienie najkrótszych ścieżek pomiędzy poszczególnymi stolikami.
				\item Problem drugi: znalezienie optymalnej kolejności stolików.
			\end{enumerate}
			Rozwiązaniem jest zaimplementowanie dwóch przestrzeni stanów (na różnym poziomie abstrakcji).
		\end{frame}
		\begin{frame}{Pierwsza przestrzeń stanów}
			Pierwsza przestrzeń stanów będzie odpowiadać za odnajdywanie najkrótszej ścieżki pomiędzy dwoma stolikami.
			\begin{enumerate}
				\item Stany - pojedyncze pola, na których może stać kelner.
				\item Przejścia pomiędzy stanami - przemieszczenie się kelnera na sąsiednie pole. Zakładamy, że każde takie przejście ma taki sam koszt.
			\end{enumerate}
			Rozwiązujemy problem najkrótszej ścieżki w grafie.
		\end{frame}
		\begin{frame}{Algorytm znajdowania najkrótszej ścieżki}
			Definiujemy $3$ zbiory pól:
			\begin{itemize}
				\item Sprawdzone
				\item DoSprawdzenia
				\item KolejneDoSprawdzenia
			\end{itemize}
			Wybieramy pole początkowe. Zmiennej $dystans$ przypisujemy wartość $0$.
		\end{frame}
		\begin{frame}{Algorytm znajdowania najkrótszej ścieżki}
			\begin{alg}
				\begin{enumerate}
					\item Dodajemy pole poczętkowe do zbioru DoSprawdzenia
					\item Przypisujemy mu wartość $0$
					\item Dopóki zbiór DoSprawdzenia nie jest pusty:
					\begin{itemize}
						\item zwiększamy zmienną $dystans$ o $1$
						\item wszystkim polom, które nie należą do zbiorów Sprawdzone lub DoSprawdzenia, a które sąsiadują z polami ze zbioru DoSprawdzenia przypisujemy wartość $dystans$
						\item dodajemy te pola do zbioru KolejneDoSprawdzenia
						\item do zbioru Sprawdzone dodajemy pola ze zbioru DoSprawdzenia
						\item zbiór DoSprawdzenia zastępujemy zbiorem KolejneDoSprawdzenia
						\item usuwamy wszystkie elementy ze zbioru KolejneDoSprawdzenia
					\end{itemize}
				\end{enumerate}
			\end{alg}
		\end{frame}
		\begin{frame}{Kod algorytmu przypisywania odległości}
			\pgfdeclareimage[width=9cm,height=7cm]{BP}{kody/kod_odleglosci.png}
			\pgfuseimage{BP}
		\end{frame}
		\begin{frame}{Przebieg działania algorytmu - przypisywanie odległości}
			\pgfdeclareimage[width=6cm,height=6cm]{BP}{kroki/krok0.png}
			\pgfuseimage{BP}
		\end{frame}
		\begin{frame}{Przebieg działania algorytmu - przypisywanie odległości}
			\pgfdeclareimage[width=6cm,height=6cm]{BP}{kroki/krok1.png}
			\pgfuseimage{BP}
		\end{frame}
		\begin{frame}{Przebieg działania algorytmu - przypisywanie odległości}
			\pgfdeclareimage[width=6cm,height=6cm]{BP}{kroki/krok2.png}
			\pgfuseimage{BP}
		\end{frame}
		\begin{frame}{Przebieg działania algorytmu - przypisywanie odległości}
			\pgfdeclareimage[width=6cm,height=6cm]{BP}{kroki/krok3.png}
			\pgfuseimage{BP}
		\end{frame}
		\begin{frame}{Przebieg działania algorytmu - przypisywanie odległości}
			\pgfdeclareimage[width=6cm,height=6cm]{BP}{kroki/krok4.png}
			\pgfuseimage{BP}
		\end{frame}
		\begin{frame}{Przebieg działania algorytmu - przypisywanie odległości}
			\pgfdeclareimage[width=6cm,height=6cm]{BP}{kroki/krok5.png}
			\pgfuseimage{BP}
		\end{frame}
		\begin{frame}{Przebieg działania algorytmu - przypisywanie odległości}
			\pgfdeclareimage[width=6cm,height=6cm]{BP}{kroki/krok6.png}
			\pgfuseimage{BP}
		\end{frame}
		\begin{frame}{Przebieg działania algorytmu - przypisywanie odległości}
			\pgfdeclareimage[width=6cm,height=6cm]{BP}{kroki/krok7.png}
			\pgfuseimage{BP}
		\end{frame}
		\begin{frame}{Przebieg działania algorytmu - przypisywanie odległości}
			\pgfdeclareimage[width=6cm,height=6cm]{BP}{kroki/krok8.png}
			\pgfuseimage{BP}
		\end{frame}
		\begin{frame}{Przebieg działania algorytmu - przypisywanie odległości}
			\pgfdeclareimage[width=6cm,height=6cm]{BP}{kroki/krok9.png}
			\pgfuseimage{BP}
		\end{frame}
		\begin{frame}{Przebieg działania algorytmu - przypisywanie odległości}
			\pgfdeclareimage[width=6cm,height=6cm]{BP}{kroki/krok10.png}
			\pgfuseimage{BP}
		\end{frame}
		\begin{frame}{Przebieg działania algorytmu - przypisywanie odległości}
			\pgfdeclareimage[width=6cm,height=6cm]{BP}{kroki/krok11.png}
			\pgfuseimage{BP}
		\end{frame}
		\begin{frame}{Przebieg działania algorytmu - przypisywanie odległości}
			\pgfdeclareimage[width=6cm,height=6cm]{BP}{kroki/krok12.png}
			\pgfuseimage{BP}
		\end{frame}
		\begin{frame}{Przebieg działania algorytmu - przypisywanie odległości}
			\pgfdeclareimage[width=6cm,height=6cm]{BP}{kroki/krok13.png}
			\pgfuseimage{BP}
		\end{frame}
		\begin{frame}{Algorytm znajdowania najkrótszej ścieżki}
			Aby odczytać najkrótszą ścieżkę z pola początkowego do wybranego pola wykonujemy następującą procedurę:
			\begin{alg}
				\begin{enumerate}
					\item Odczytujemy wartość wybranego pola
					\item Dodajemy wybrane pole do ścieżki
					\item Dla $i$ od wartości wybranego pola do $1$:
					\begin{itemize}
						\item ustawiamy bierzące pole na sąsiednie pole o wartości $i-1$
						\item dopisujemy bierzące pole do ścieżki
					\end{itemize}
					\item Odwracamy kolejność pól na ścieżce
				\end{enumerate}
			\end{alg}
		\end{frame}
		\begin{frame}{Kod algorytmu odnajdowania ścieżki}
			\pgfdeclareimage[width=9cm,height=7cm]{BP}{kody/kod_sciezka.png}
			\pgfuseimage{BP}
		\end{frame}
		\begin{frame}{Przebieg działania algorytmu - odnajdowanie ścieżki}
			\pgfdeclareimage[width=6cm,height=6cm]{BP}{kroki/krok14.png}
			\pgfuseimage{BP}
		\end{frame}
		\begin{frame}{Przebieg działania algorytmu - odnajdowanie ścieżki}
			\pgfdeclareimage[width=6cm,height=6cm]{BP}{kroki/krok15.png}
			\pgfuseimage{BP}
		\end{frame}
		\begin{frame}{Przebieg działania algorytmu - odnajdowanie ścieżki}
			\pgfdeclareimage[width=6cm,height=6cm]{BP}{kroki/krok16.png}
			\pgfuseimage{BP}
		\end{frame}
		\begin{frame}{Przebieg działania algorytmu - odnajdowanie ścieżki}
			\pgfdeclareimage[width=6cm,height=6cm]{BP}{kroki/krok17.png}
			\pgfuseimage{BP}
		\end{frame}
		\begin{frame}{Przebieg działania algorytmu - odnajdowanie ścieżki}
			\pgfdeclareimage[width=6cm,height=6cm]{BP}{kroki/krok18.png}
			\pgfuseimage{BP}
		\end{frame}
		\begin{frame}{Przebieg działania algorytmu - odnajdowanie ścieżki}
			\pgfdeclareimage[width=6cm,height=6cm]{BP}{kroki/krok19.png}
			\pgfuseimage{BP}
		\end{frame}
		\begin{frame}{Przebieg działania algorytmu - odnajdowanie ścieżki}
			\pgfdeclareimage[width=6cm,height=6cm]{BP}{kroki/krok20.png}
			\pgfuseimage{BP}
		\end{frame}
		\begin{frame}{Pierwsza przestszeń stanów - wielkość danych}
			Dla $n$ stolików mamy $n+1$ wyróżnionych pól (łącznie z polem w lewym górnym rogu mapy). Mamy zatem $n \times (n+1) = n^2 + n$ wszystkich ścieżek. Tak obliczone ścieżki mogą być wielokrotnie wykorzystywane. Możemy zatem dokonać obliczenia wszstkich najkrótszych ścieżek pomiędzy wyróżnionymi polami na samym początku działania programu (w ramach obliczeń wstępnych).
		\end{frame}
		\begin{frame}{Druga przestrzeń stanów}
			Druga przestrzeń stanów będzie odpowiadać za znajdowanie najlepszej kolejności stolików.
			\begin{enumerate}
				\item Stany - pola przy stolikach, do których powinien podejść kelner oraz pole startowe w lewym górnym rogu mapy.
				\item Przejścia pomiędzy stanami - najkrótsze ścieżki pomiędzy stolikami, które odwiedza kelner. Koszt takiego przejścia jest równy długości ścieżki.
			\end{enumerate}
			Rozwiązujemy problem komiwojażera w grafie pełnym.
		\end{frame}
		\begin{frame}{Problem komiwojażera}
			Problem komiwojażera jest problemem NP-trudnym. Oznacza to, że nie jest znany algorytm wielomianowy, który go rozwiązuje. Istnieją jednak metody działające w czasie wielomianowym, które dają wynik zbliżony do optymalnego. W naszej metodzie wykorzystamy algorytm Prima, służący do wyznaczania minimalnego drzewa rozpiętego w grafie.
		\end{frame}
		\begin{frame}{Algorytm Prima}
			\begin{alg}
				\begin{enumerate}
					\item Tworzymy drzewo zawierające pojedynczy wierzchołek
					\item Spośród wszystkich krawędzi wychodzących z naszego drzewa wybieramy tą o najmniejszej wadze i dołączamy ją do drzewa (wraz z wierzchołkiem leżącym na końcu tej krawędzi)
					\item Poprzedni krok powtarzamy tak długo, aż wszystkie wierzchołki grafu nie znajdą się w drzewie
				\end{enumerate}
			\end{alg}
		\end{frame}	
		\begin{frame}{Przebieg działania algorytmu Prima}
			\pgfdeclareimage[width=6cm,height=6cm]{BP}{kroczki/graf0.png}
			\pgfuseimage{BP}
		\end{frame}	
		\begin{frame}{Przebieg działania algorytmu Prima}
			\pgfdeclareimage[width=6cm,height=6cm]{BP}{kroczki/graf1.png}
			\pgfuseimage{BP}
		\end{frame}	
		\begin{frame}{Przebieg działania algorytmu Prima}
			\pgfdeclareimage[width=6cm,height=6cm]{BP}{kroczki/graf2.png}
			\pgfuseimage{BP}
		\end{frame}	
		\begin{frame}{Przebieg działania algorytmu Prima}
			\pgfdeclareimage[width=6cm,height=6cm]{BP}{kroczki/graf3.png}
			\pgfuseimage{BP}
		\end{frame}	
		\begin{frame}{Przebieg działania algorytmu Prima}
			\pgfdeclareimage[width=6cm,height=6cm]{BP}{kroczki/graf4.png}
			\pgfuseimage{BP}
		\end{frame}	
		\begin{frame}{Przebieg działania algorytmu Prima}
			\pgfdeclareimage[width=6cm,height=6cm]{BP}{kroczki/graf5.png}
			\pgfuseimage{BP}
		\end{frame}	
		\begin{frame}{Przebieg działania algorytmu Prima}
			\pgfdeclareimage[width=6cm,height=6cm]{BP}{kroczki/graf6.png}
			\pgfuseimage{BP}
		\end{frame}	
		\begin{frame}{Przebieg działania algorytmu Prima}
			\pgfdeclareimage[width=6cm,height=6cm]{BP}{kroczki/graf7.png}
			\pgfuseimage{BP}
		\end{frame}	
		\begin{frame}{Przebieg działania algorytmu Prima}
			\pgfdeclareimage[width=6cm,height=6cm]{BP}{kroczki/graf8.png}
			\pgfuseimage{BP}
		\end{frame}	
		\begin{frame}{Przebieg działania algorytmu Prima}
			\pgfdeclareimage[width=6cm,height=6cm]{BP}{kroczki/graf9.png}
			\pgfuseimage{BP}
		\end{frame}	
		\begin{frame}{Przebieg działania algorytmu Prima}
			\pgfdeclareimage[width=6cm,height=6cm]{BP}{kroczki/graf10.png}
			\pgfuseimage{BP}
		\end{frame}	
		\begin{frame}{Przebieg działania algorytmu Prima}
			\pgfdeclareimage[width=6cm,height=6cm]{BP}{kroczki/graf11.png}
			\pgfuseimage{BP}
		\end{frame}
		\begin{frame}{Minimalne drzewo rozpięte}
			\begin{uw}
				Załóżmy, że wagi krawędzi w naszym grafie są dodatnie. Zauważmy, że jeżeli cykl $\{v_0, v_1\} , \{v_1, v_2\} , ... , \{v_{n-1}, v_n\} , \{v_n, v_0\}$ jest rozwiązaniem problemu komiwojażera, to ścieżka $\{v_0, v_1\} , \{v_1, v_2\} , ... , \{v_{n-1}, v_n\}$ jest drzewem rozpiętym. Zatem minimalne drzewo rozpięte ma mniejszą sumę wag krawędzi niż rozwiązanie problemu komiwojażera.
			\end{uw}
		\end{frame}
		\begin{frame}{Rozwiązanie przybliżone}
			Rozpatrzmy następujące drzewo rozpięte:
			\pgfdeclareimage[width=6cm,height=6cm]{BP}{kroczki/drzewo0.png}
			\pgfuseimage{BP}
		\end{frame}
		\begin{frame}{Rozwiązanie przybliżone}
			Przechodzimy przez całe drzewo i wracamy do wierzchołka $0$ (każdą krawędź przechodzimy dwa razy).
			\pgfdeclareimage[width=6cm,height=6cm]{BP}{kroczki/drzewo1.png}
			\pgfuseimage{BP}
		\end{frame}
		\begin{frame}{Rozwiązanie przybliżone}
			Pomijamy powtarzające się wierzchołki (skracamy drogę).
			\pgfdeclareimage[width=6cm,height=6cm]{BP}{kroczki/drzewo2.png}
			\pgfuseimage{BP}
		\end{frame}
		\begin{frame}{Rozwiązanie przybliżone}
			Otrzymaliśmy rozwiązanie co najwyżej dwa razy gorsze od optymalnego.
			\pgfdeclareimage[width=6cm,height=6cm]{BP}{kroczki/drzewo2.png}
			\pgfuseimage{BP}
		\end{frame}
		\begin{frame}{Kod algorytmu wyznaczania kolejności stolików}
			\pgfdeclareimage[width=11cm,height=7cm]{BP}{kody/kod_kolejnosc.png}
			\pgfuseimage{BP}
		\end{frame}
\end{document}