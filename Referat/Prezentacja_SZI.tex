\documentclass[a4paper,10pt]{beamer}
\usepackage[utf8]{inputenc}
\usepackage{polski}
\usepackage[OT4,T1]{fontenc}
\usepackage{amsmath}
\usepackage{amsthm}
\usepackage{graphicx}
\usepackage{ dsfont }
\usepackage{ amssymb }
\usepackage{enumerate}
\usepackage{tikz}

\usetheme{Warsaw}
\usecolortheme{beaver}

\newtheorem{defi}{Definicja}[subsection]
\newtheorem{uw}{Uwaga}[subsection]
\newtheorem{cel}{Cel}[subsection]
\newtheorem{tw}{Twierdzenie}[subsection]
\newtheorem{lem}{Lemat}[subsection]
\newtheorem{przyk}{Przykład}[subsection]
\newtheorem{alg}{Algorytm}[subsection]

\date{15 czerwca 2015}
\title{Inteligentny kelner}
\author{Andrzej Bochonos, Dominik Demski, Agnieszka Limiszewska, Adam Mieldzioc}

\begin{document}
		\begin{frame}
			\titlepage
		\end{frame}
		\begin{frame}{Agenda}
			\tableofcontents
		\end{frame}
		
		\section{Algorytmy genetyczne}
		\begin{frame}{Do czego zastosowałem algorytmy genetyczne?}
			\begin{defi}
				Kombiwojażer musi odwiedzić wszystkie miasta z zadanego regionu i wrócić do miasta początkowego (jest to problem szukania cyklu). Wszystkie miasta są ze sobą połączone (mamy do czynienia z grafem pełnym). Mając do dyspozycji macierz odległości pomiędzy poszczególnymi miastami, należy znaleść cykl o najmniejszym koszcie, przy czym każde miasto nie może być odwiedzone więcej niż jeden raz.
			\end{defi}
			
		\end{frame}
		\begin{frame}{Problem kombiwojażera w aplikacji}
			\begin{enumerate}
				\item miasta = stoły
				\item kombiwojażer = kelner
			\end{enumerate}
		\end{frame}
		\begin{frame}{Algorytm}
			\begin{alg}
								
			\end{alg}
		\end{frame}
		
\end{document}