\documentclass[a4paper,10pt]{beamer}
\usepackage[utf8]{inputenc}
\usepackage{polski}
\usepackage[OT4,T1]{fontenc}
\usepackage{amsmath}
\usepackage{amsthm}
\usepackage{graphicx}
\usepackage{ dsfont }
\usepackage{ amssymb }
\usepackage{enumerate}
\usepackage{tikz}

\usetheme{Warsaw}
\usecolortheme{beaver}

\newtheorem{defi}{Definicja}[subsection]
\newtheorem{uw}{Uwaga}[subsection]
\newtheorem{cel}{Cel}[subsection]
\newtheorem{tw}{Twierdzenie}[subsection]
\newtheorem{lem}{Lemat}[subsection]
\newtheorem{przyk}{Przykład}[subsection]
\newtheorem{alg}{Algorytm}[subsection]

\date{15 czerwca 2015}
\title{Inteligentny kelner}
\author{Andrzej Bochonos, Dominik Demski, Agnieszka Limiszewska, Adam Mieldzioc}

\begin{document}
		\begin{frame}
			\titlepage
		\end{frame}
		\begin{frame}{Agenda}
			\tableofcontents
		\end{frame}
		
		\section{Algorytmy genetyczne}
		\begin{frame}{Do czego zastosowałem algorytmy genetyczne?}
			\begin{defi}
				Kombiwojażer musi odwiedzić wszystkie miasta z zadanego regionu i wrócić do miasta początkowego (jest to problem szukania cyklu). Wszystkie miasta są ze sobą połączone (mamy do czynienia z grafem pełnym). Mając do dyspozycji macierz odległości pomiędzy poszczególnymi miastami, należy znaleść cykl o najmniejszym koszcie, przy czym każde miasto nie może być odwiedzone więcej niż jeden raz.
			\end{defi}
			
		\end{frame}
		\begin{frame}{Problem kombiwojażera w aplikacji}
			\begin{enumerate}
				\item miasta = stoły
				\item kombiwojażer = kelner
			\end{enumerate}
		\end{frame}
		\begin{frame}{Kolejne elementy algorytmu genetycznego}
			\begin{alg}
				\begin{enumerate}
					\item Populacja
					\item Ocena osobników
					\item Selekcja
					\item Krzyżowanie
					\item Mutacja
				\end{enumerate}
			\end{alg}
		\end{frame}
		\begin{frame}{Populacja}
			\begin{enumerate}
				\item Rozmiar populacji: 1000
				\item Ilość ewolucji: 2000
			\end{enumerate}
		\end{frame}
		\begin{frame}{Reprezentacja chromosomu}
			Pojedynczy chromosom ma reprezentować proponowaną kolejność odwiedzania stolików. Chromosom reprezentuję jako listę pokazująca kolejność pobierania stolików do tworzonej trasy.
			
			\begin{przyk}
					Punktem odniesienia dla tej reprezentacji jest lista kolejnych stolików: 1-2-3-4-5. Pojedynczy osobnik np. 3-3-0-1-0 pokazuje w jakiej kolejności wybierane są kolejno odwiedzane stoliki. Na początku jest trójka więc pierwszym odwiedzanym stolikiem będzie stolik umieszczony na trzeciej pozycji w liście odniesienia, czyli czwórka. Czwórkę tę usuwa się z listy odniesienia (pozostają stoliki 1-2-3-5), natomiast lista odwiedzanych stolików wygląda następująco: 4.
					Kolejnym elementem osobnika jest ponownie trójka. W tej chwili na trzecim miejscu listy odniesienia jest piątka, więc kolejnym odwiedzanym stolikiem będzie stolik nr 5, a lista odniesienia będzie wyglądała następująco: 1-2-3 itd.
			\end{przyk}
		\end{frame}
		\begin{frame}{Wady i zalety takiej reprezentacji}
			Reprezentacja ta wprowadza spore zamieszanie przy przechodzeniu na reprezentację wykorzystywaną przy funcji oceny, jednak kłopoty te rekompensuje przy krzyżowaniu i mutacji. Cechą charakterystyczą tej reprezentacji jest fakt, że na i-tej pozycji jest liczba z przedziału od 0 do n-i-1 (gzie n to liczba wszystkich stolików, np. na miejscu zerowym wszystkie miejsca są jeszcze do wybrania, jest więc n - 0 możliwości, należy odjąć 1, bo lista numerowana jest od zera). Ze względu na to wymiana materiału genetycznego między dwoma osobnikami za pomocą standardowego krzyżowania x-punktowego zawsze da dopuszczalne potomstwo.
		\end{frame}
		\begin{frame}{Kod}
			\pgfdeclareimage[width=10cm,height=8cm]{chromosom}{przykladowyChromosom.png}
			\pgfuseimage{chromosom}
		\end{frame}
\end{document}