\documentclass[a4paper,10pt]{beamer}
\usepackage[utf8]{inputenc}
\usepackage{polski}
\usepackage[OT4,T1]{fontenc}
\usepackage{amsmath}
\usepackage{amsthm}
\usepackage{graphicx}
\usepackage{ dsfont }
\usepackage{ amssymb }
\usepackage{enumerate}
\usepackage{tikz}

\usetheme{Warsaw}
\usecolortheme{beaver}

\newtheorem{defi}{Definicja}[subsection]
\newtheorem{uw}{Uwaga}[subsection]
\newtheorem{cel}{Cel}[subsection]
\newtheorem{tw}{Twierdzenie}[subsection]
\newtheorem{lem}{Lemat}[subsection]
\newtheorem{przyk}{Przykład}[subsection]
\newtheorem{alg}{Algorytm}[subsection]

\date{15 czerwca 2015}
\title{Inteligentny kelner}
\author[A. Bohonos, D. Demski, A. Limiszewska, A. Mieldzioc]{Andrzej Bohonos, Dominik Demski, Agnieszka Limiszewska, Adam Mieldzioc}

\begin{document}
		\begin{frame}
			\titlepage
		\end{frame}
		\begin{frame}{Agenda}
			\tableofcontents
		\end{frame}
		
		\section{Algorytmy genetyczne}
		\begin{frame}{Do czego zastosowałem algorytmy genetyczne?}
			Skorzystałem z biblioteki JGAP.
			\begin{defi}
				Kombiwojażer musi odwiedzić wszystkie miasta z zadanego regionu i wrócić do miasta początkowego (jest to problem szukania cyklu). Wszystkie miasta są ze sobą połączone (mamy do czynienia z grafem pełnym). Mając do dyspozycji macierz odległości pomiędzy poszczególnymi miastami, należy znaleść cykl o najmniejszym koszcie, przy czym każde miasto nie może być odwiedzone więcej niż jeden raz.
			\end{defi}
			
		\end{frame}
		\begin{frame}{Problem kombiwojażera w aplikacji}
			\begin{enumerate}
				\item miasta = stoły
				\item kombiwojażer = kelner
			\end{enumerate}
		\end{frame}
		\begin{frame}{Kolejne elementy algorytmu genetycznego}
			\begin{alg}
				\begin{enumerate}
					\item Populacja
					\item Krzyżowanie
					\item Mutacja
					\item Ocena osobników
					\item Selekcja
				\end{enumerate}
			\end{alg}
		\end{frame}
		\begin{frame}{Populacja}
			\begin{enumerate}
				\item Rozmiar populacji: 1000
				\item Ilość ewolucji: 2000
				\item Na początku osobniki tworzone są w sposób losowy - na podstawie przykładowego chromosomu.
			\end{enumerate}
		\end{frame}
		\begin{frame}{Reprezentacja chromosomu}
			Pojedynczy chromosom ma reprezentować proponowaną kolejność odwiedzania stolików. Chromosom reprezentuję jako listę pokazująca kolejność pobierania stolików do tworzonej trasy.
			
			\begin{przyk}
					Punktem odniesienia dla tej reprezentacji jest lista kolejnych stolików: 1-2-3-4-5. Pojedynczy osobnik np. 3-3-0-1-0 pokazuje w jakiej kolejności wybierane są kolejno odwiedzane stoliki. Na początku jest trójka więc pierwszym odwiedzanym stolikiem będzie stolik umieszczony na trzeciej pozycji w liście odniesienia, czyli czwórka. Czwórkę tę usuwa się z listy odniesienia (pozostają stoliki 1-2-3-5), natomiast lista odwiedzanych stolików wygląda następująco: 4.
					Kolejnym elementem osobnika jest ponownie trójka. W tej chwili na trzecim miejscu listy odniesienia jest piątka, więc kolejnym odwiedzanym stolikiem będzie stolik nr 5, a lista odniesienia będzie wyglądała następująco: 1-2-3 itd.
			\end{przyk}
		\end{frame}
		\begin{frame}{Wady i zalety takiej reprezentacji}
			Reprezentacja ta wprowadza spore zamieszanie przy przechodzeniu na reprezentację wykorzystywaną przy funcji oceny, jednak kłopoty te rekompensuje przy krzyżowaniu i mutacji. Cechą charakterystyczą tej reprezentacji jest fakt, że na i-tej pozycji jest liczba z przedziału od 0 do n-i-1 (gzie n to liczba wszystkich stolików, np. na miejscu zerowym wszystkie miejsca są jeszcze do wybrania, jest więc n - 0 możliwości, należy odjąć 1, bo lista numerowana jest od zera). Ze względu na to wymiana materiału genetycznego między dwoma osobnikami za pomocą standardowego krzyżowania x-punktowego zawsze da dopuszczalne potomstwo.
		\end{frame}
		\begin{frame}{Kod}
			\pgfdeclareimage[width=10cm,height=8cm]{chromosom}{przykladowyChromosom.png}
			\pgfuseimage{chromosom}
		\end{frame}
		\begin{frame}{Kod}
			\pgfdeclareimage[width=9cm,height=6cm]{chromosom}{zamianaNaChromosom.png}
			\pgfuseimage{chromosom}
		\end{frame}
	
		\begin{frame}{Krzyżowanie}
			35 procent populacji bierze udział w krzyżowaniu - każde krzyżowanie daje dwa nowe osobniki. Zastosowałem krzyżowanie jednopunktowe, punkt krzyżowania wybierany jest w sposób losowy. 
		\end{frame}
		\begin{frame}{Mutacja}
			Mutacja występuje, jesli wylosowana z przedziału [0, 12] liczba całkowita jest zerem, co daje prawdopodobieństwo mutacji 1/12. Taka operacja jest stosowana do każdego genu w każdym chromosomie - z wyjątkiem tych, które dopiero co powstały w wyniku krzyżowania.
		\end{frame}
			\begin{frame}{Ocena osobników - jak mierzę odległość między stolikami?}
				Wykorzystuję odległość miejską - bo tak mniej więcej będzie chodził kelner:
				\begin{defi}
					Odległość miejska (zwana również odległością taksówkową lub manhatańską):  
					\begin{equation}
					\rho (\textbf{x}_{r}, \textbf{x}_s) = \sum\limits_{i = 1}^{p} |(x_{ri} - x_{si})|
					\end{equation}
				\end{defi}
			\end{frame}
			\begin{frame}
				\pgfdeclareimage[width=11cm,height=8cm]{chromosom}{mapa.png}
				\pgfuseimage{chromosom}
			\end{frame}
			\begin{frame}
				\pgfdeclareimage[width=11cm,height=8cm]{chromosom}{macierzOdleglosci.png}
				\pgfuseimage{chromosom}
			\end{frame}
			\begin{frame}{Ocena osobników}
				\pgfdeclareimage[width=10cm,height=8cm]{chromosom}{funkcjaDopasowania.png}
				\pgfuseimage{chromosom}
			\end{frame}
		\begin{frame}{Selekcja}
			Zastosowałem domyślną implementację (domyślnie ustawiana jest klasa BestChromosomesSelector): wybranych zostaje 90 procent najlepsszych osobników, pozostałe 10 procent uzupełnia się, kopiując najlepszych (pod względem funckji dopasowania) osobników aż do wyczerpania miejsca.
		\end{frame}
			\begin{frame}{Przykład działania}
				W ramach eksperymentu stworzyłem sobie następującą sytuację:
				\begin{enumerate}
					\item Należy, wyruszając z punktu 0, odwiedzić wszystkie stoły numerowane od 1 do 5 włącznie i wrócić do punktu 0 (każdy stół musi być odwiedzony tylko raz). 
					\item Odległość między dwoma stołami to wartość bezwzględna z różnicy ich numerów (np. odległość pomiędzy stołem 1., a 3. = |1 - 3| = 2), przy czym odległość pomiędzy stołem numer 5 i punktem 0 wynosi 1.
					\item Przy tak określonym zadaniu nietrudno sprawdzić, że najkrótsza droga to: [1, 2, 3, 4, 5] albo równoważna jej [5, 4, 3, 2, 1] i jej długość wynosi: 6.
					\item Uruchomię mój algorytm 10 razy z takimi samymi ustawieniami, jakie są zaimplementowane w aplikacji Inteligentny kelner (rozmiar populacji - 1000, 2000 ewolucji).
				\end{enumerate}
			\end{frame}
		\begin{frame}{Wyniki doświadczenia}
			\pgfdeclareimage[width=12cm,height=7cm]{chromosom}{eksperymentDD.png}
			\pgfuseimage{chromosom}
		\end{frame}
		\begin{frame}{Wyniki doświadczenia - ciąg dalszy}
			Dla stu powtórzeń:
			\begin{enumerate}
				\item średnia długości najlepszych znalezionych ścieżek wyniosła: 12.32
				\item najkrótsza znaleziona ścieżka miała długość: 6 (taką ścieżkę znaleziono 2 razy)
				\item najdłuższa znaleziona ścieżka miała długość: 18 (taką ścieżkę znaleziono raz)
			\end{enumerate}
		\end{frame}
		\begin{frame}{Wyniki doświadczenia - 1000 powtórzeń}
			Algorytm wykonywał się 1, 5 godziny.
			\pgfdeclareimage[width=12cm,height=7cm]{chromosom}{eksperyment1000.png}
			\pgfuseimage{chromosom}
		\end{frame}
\end{document}